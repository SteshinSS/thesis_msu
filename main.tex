\documentclass[a4paper, 12pt, left=30mm, right=15mm, top=20mm, bottom=20mm]{report}
\usepackage[utf8]{inputenc}
\usepackage[russian]{babel}
\usepackage{mhchem} % for 13C
\usepackage{verbatim} % for multiline commentary 
\usepackage{graphicx} % for logo
\usepackage{titling} % for shift margins
\usepackage{blindtext} % for text-generation
\usepackage{setspace} % to set interval


\begin{document}

\begin{titlepage}

	\begin{centering}
		\includegraphics[width=0.25\textwidth]{logo.png}\par
	\end{centering}
	\centerline{Московский государственный университет имени М. В. Ломоносова}
	\centerline{Факультет вычислительной математики и кибернетики}
	\centerline{Кафедра математической кибернетики}
	\centerline{\hfill\hrulefill\hrulefill\hfill}
	\vfill
	\vfill
	\large
	\centerline{Стешин Семен Сергеевич}
	\vfill
	\Large
	\begin{centering}
		\textbf{Khnum: быстрая open-source программа \\ для расчета метаболических потоков \\ с использованием \ce{^{13}C}-углерода}
		
	\end{centering}
	\normalsize
	\vfill
	\centerline{Выпускная квалификационная работа}
	\vfill
	\vfill
	\begin{flushright}
	Научный руководитель:\\
	к.ф.м.н. М.С.Шуплецов
	\end{flushright}
	\vfill
	\vfill
	\centerline{Москва --- 2020}

\end{titlepage}



\begin{abstract}
\blindtext
\end{abstract}

\tableofcontents

\chapter{Введение}
\section{Популярное введение}
\blindtext

\chapter{Основные понятия}
\section{Глоссарий}
\section{Прямая задача}
\section{Обратная задача}
\chapter{Постановка задачи}
\chapter{Основная часть}
\chapter{Полученные результаты}

\begin{thebibliography}{XXXX}
	\bibitem[SHU1]{Shupletsov1}
	OpenFLUX2: 13C-MFA modeling software package adjusted for the comprehensive analysis of single and parallel labeling experiments. Shupletsov
\end{thebibliography}

\end{document}          
