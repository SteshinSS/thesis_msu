\documentclass[a4paper, 12pt, left=30mm, right=15mm, top=20mm, bottom=20mm]{report}
\usepackage[utf8]{inputenc}
\usepackage[russian]{babel}
\usepackage{mhchem} % for 13C
\usepackage{verbatim} % for multiline commentary 
\usepackage{graphicx} % for logo
\usepackage{titling} % for shift margins
\usepackage{blindtext} % for text-generation
\usepackage{setspace} % to set interval


\begin{document}

\begin{titlepage}

	\begin{centering}
		\includegraphics[width=0.25\textwidth]{logo.png}\par
	\end{centering}
	\centerline{Московский государственный университет имени М. В. Ломоносова}
	\centerline{Факультет вычислительной математики и кибернетики}
	\centerline{Кафедра математической кибернетики}
	\centerline{\hfill\hrulefill\hrulefill\hfill}
	\vfill
	\vfill
	\large
	\centerline{Стешин Семен Сергеевич}
	\vfill
	\Large
	\begin{centering}
		\textbf{Khnum: быстрая open-source программа \\ для расчета метаболических потоков \\ с использованием \ce{^{13}C}-углерода}
		
	\end{centering}
	\normalsize
	\vfill
	\centerline{Выпускная квалификационная работа}
	\vfill
	\vfill
	\begin{flushright}
	Научный руководитель:\\
	к.ф.м.н. М. С. Шуплецов
	\end{flushright}
	\vfill
	\vfill
	\centerline{Москва --- 2020}

\end{titlepage}



\begin{abstract}
В биологии и медицине встречается задача определения скорости метаболических потоков внутри клетки. 
Один из методов решения этой задачи --- анализ метаболических потоков с использованием \ce{^{13}C}-углерода (\ce{^{13}C}-Metabolic Flux Analysis). 
В этом методе, исследователи проводят эксперимент и обрабатывают его результаты на компьютере. Проблема в том, что современные программы для анализа метаболических потоков либо имеют закрытый код и платны для коммерческого использования, либо написаны неэффективно, из-за чего вычисления могут занимать недели для одного эксперимента. В этой работе проведен краткий обзор метода, написана эффективная программа для решения задачи и проведено сравнение с существующими аналогами.
\end{abstract}

\tableofcontents

\chapter{Введение}
\section{Мотивация}
Рак --- вторая по частоте причина смерти в мире\cite{Cancer_statistics}. Сто лет назад Отто Варбург заметил\cite{Warburg_effect} особенность раковых клеток: они склонны производить энергию с помощью активного гликолиза, вместо более эффективного окислительного фосфорилирования. Знание этого позволило находить опухоли с помощью позитронно-эмиссионной томографии, а Варбурга наградили Нобелевской премией.

Диабетом болеет 8.8\% людей в мире\cite{Diabetes_statistics}. Почти 4 миллиона в год умирает из-за этой болезни. Лечения пока нет, но есть симптоматическая терапия инъекциями инсулина. Раньше его получали из поджелудочных желез свиней и коров, но препарат было сложно очистить, поэтому иногда случались аллергические реакции. Все изменилось в 1978 году, когда компания Genentech смогла создать генетически-модифицированную кишечную палочку, которая в ходе жизнедеятельности производила чистый человеческий инсулин\cite{Genentech_paper}. Сейчас таким образом производят почти весь препарат.

В случае с эффектом Варбурга, открытие заключалось в изменении скорости химической реакции, протекающей внутри клетки. В случае с инсулином, решается задача метаболической инженерии --- увеличить скорость синтеза инсулина, не убив кишечную палочку. В обоих случаях надо уметь измерять скорости внутриклеточных химических реакций -- их называют потоками. Один из современных методов измерения потоков -- \emph{\ce{^{13}C}-Metabolic Flux Analysis} (далее \emph{MFA}), что переводится как анализ метаболических потоков. Этому методу посвящена наша работа.

\section{Эксперимент}
\section{Компьютерное моделирование}

\chapter{Основные понятия}
\section{Глоссарий}
\section{Прямая задача}
\section{Обратная задача}
\chapter{Постановка задачи}
\chapter{Основная часть}
\chapter{Полученные результаты}

\begin{thebibliography}{XXXX}
	\bibitem{Cancer_statistics}
	Всемирная Ассоциация Здравоохранения. Cancer [Электронный ресурс] URL: https://www.who.int/news-room/fact-sheets/detail/cancer (дата обращения: 12.03.2020)
	
	\bibitem{Warburg_effect}
	Warburg O., Wind F., Negelein E. The metabolism of tumors in the body //The Journal of general physiology. – 1927. – Т. 8. – №. 6. – С. 519.
	
	\bibitem{Diabetes_statistics}
	Zimmet P. et al. Diabetes mellitus statistics on prevalence and mortality: facts and fallacies //Nature Reviews Endocrinology. – 2016. – Т. 12. – №. 10. – С. 616.
	
	\bibitem{Genentech_paper}
	Cohen S. N. et al. Construction of biologically functional bacterial plasmids in vitro //Proceedings of the National Academy of Sciences. – 1973. – Т. 70. – №. 11. – С. 3240-3244.
\end{thebibliography}

\end{document}          
